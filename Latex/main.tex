\documentclass{article}
\usepackage[utf8]{inputenc}

\title{TRTP-Project}
\author{Kouider Ben Naoum}
\date{September 2019}

\begin{document}

\maketitle
%CONTENU A VOIR
%- Comment maintenir l'état des connections concurrentes
%- Quelle est notre technique pour traiter les connections concurrentes
%- Comment avez-vous implémenté le principe de fenêtres de réception
%- Quelle est votre statégie pour la génération des acquittements?

%- Quelle est la partie critique de votre implémentation, affectant la vitesse de transfert.
%- Comment géréz-vous la fermeture de connection?
%- Quelles sont les performances de votre implémentation (Graphes)
%- Quelles stragégies de tests avez-vous utilisées

%- Test d'interopérabilité en annexe, ainsi que les changements faits au code.


\section{Introduction}
%Il faut réintroduire le projet, rappelle-toi romain que nous offrons ce rapport à
%une entreprise qui veut acheter notre solution.
\section{Implémentation}
\subsection{Selective repeat et fenêtre de connection}
%Diamond girl je t’apprécie beaucoup, j’aimerais te prendre dans mes bras mais aussi tirer mon coup

\subsection{Génération d'acquitements}
%Y'a trop rien a dire, mais je vais en gros expliquer comment c'est que le gros zozo delcoigne a fait ça

\subsection{Connections concurrentes}
%Expliquer comment c'est qu'on fait (l'array de buffers dans queue), l'array de file descriptors, l'array de addrinfo

\section{Evalutation}

\subsection{Section critique (lente) du code}
%Je ne sais même pas ce qui ralentit le code... Select? Et éventuellement l'ajout dans le buffer?
%Ou bien le fait d'écrire dans le fichier si c'est pas un SSD.
\subsection{Phase de tests}
%On explique les tests qu'on a fait

%Donc les encode decode tu dis comment tu as fait
%Test des sockets
%Test du selective repeat
\subsection{Phase de benchmarking}
%Chiffres et graphes sur la vitesse,... ?
%Idées de graphes à mettre
%- Evolution de la mémoire utilisée
%- Performances?!

\end{document}
